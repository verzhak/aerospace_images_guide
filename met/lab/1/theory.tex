
\mysubsubsection{Данные дистанционного зондирования Земли}

Для решения задач картографирования земной поверхности, оперативного мониторинга природопользования, учета природных ресурсов и многих других задач широко применяются аэро- и спутниковые снимки, сделанные различной техникой.

Аэросъемка производится с помощью специальных камер, установленных на борту летательных аппаратов, потолок высот полетов которых не превосходит 20-ти километров. Стоимость аэросъемки достаточно высока, поэтому ее применение, в большинстве случаев, не является экономически выгодным.

В настоящее время большой популярностью пользуются данные ДЗЗ, полученные различными космическими аппаратами (искусственными спутниками Земли; КА). Стоимость данных ДЗЗ варьируется в широком ценовом диапазоне и зависит от ряда факторов - количества спектральных каналов в снимках, периодичность съемки, возможность съемки на заказ, размер одного кадра снимка, способы предоработки снимков и прочих. В образовательных целях разумно использовать данные ДЗЗ, доступные бесплатно и без каких - либо ограничений.

Существует несколько десятков бесплатных продуктов, полученных по данным ДЗЗ, среди которых можно выделить следующие:

\begin{itemize}

	\item Спутниковые снимки и карты низкого, среднего и (для ряда областей земной поверхности) высокого разрешения, предоставляемые web - сервисами <<Google Maps>>, <<Яндекс.Карты>> и прочими.

	Как правило, данные сервисы предоставляют спутниковые снимки, содержащие три спектральных канала - синий, зеленый и красный - причем каждый из каналов подвергается существенной предобработке с целью повысить визуальное качество снимка и различимость объектов на нем.

	Из-за существенной предобработки спутниковых снимков, небольшого количества спектральных каналов и лицензионных ограничений, накладываемых на распространение снимков, их практическое применение затруднено;

	\item Данные ДЗЗ, полученные КА группировки Landsat. \cite{gis-lab-l5} \cite{gis-lab-l7}

	В настоящее время на орбите функционируют КА Landsat 5 и Landsat 7. Оба спутника выполняют регулярную съемку большей части земной поверхности, но не предоставляют возможности съемки на заказ.

	КА Landsat 5 выполняет съемку земной поверхности несколькими устройствами, одним из которых является многоспектральный оптико - механический сканирующий радиометр <<Thematic Mapper>> (TM). Камера TM выполняет съемку в следующих спектральных каналах:

	\newcommand{\channel}[6]
	{
		\item B{#1}0 - #2 канал.

		Разрешение: #3 метров на пиксель.
		Диапазон: #4 нм. - #5 нм.#6
	}

	\begin{itemize}

		\channel{1}{синий (B)}{30}{450}{520}{;}
		\channel{2}{зеленый (G)}{30}{520}{605}{;}
		\channel{3}{красный (R)}{30}{630}{690}{;}
		\channel{4}{ближний инфракрасный (NIR, VNIR)}{30}{760}{900}{;}
		\channel{5}{коротковолновой инфракрасный (SWIR)}{30}{1550}{1750}{;}
		\channel{6}{тепловой инфракрасный (TIR)}{120}{10400}{12500}{;}
		\channel{7}{коротковолновой инфракрасный (SWIR)}{30}{2080}{2350}{.}

	\end{itemize}
			
	КА Landsat 7 выполняет съемку земной поверхности многоспектральным оптико - механическим сканирующим радиометром <<Enhanced Thematic Mapper>> (ETM+). Камера ETM+ выполняет съемку в следующих спектральных каналах:

	\begin{itemize}

		\channel{1}{синий (B)}{30}{450}{515}{;}
		\channel{2}{зеленый (G)}{30}{520}{605}{;}
		\channel{3}{красный (R)}{30}{630}{690}{;}
		\channel{4}{ближний инфракрасный (NIR, VNIR)}{30}{760}{900}{;}
		\channel{5}{коротковолновой инфракрасный (SWIR)}{30}{1550}{1750}{;}
		\channel{6}{тепловой инфракрасный (TIR)}{120}{10400}{12500}{;}
		\channel{7}{коротковолновой инфракрасный (SWIR)}{30}{2080}{2350}{;}
		\channel{8}{панхроматический}{15}{520}{900}{.}

	\end{itemize}

	Наличие панхроматического канала с разрешением, вдвое большим чем разрешения прочих каналов, суть несомненное достоинство камеры ETM+.

	К сожалению, в 2003-ем году на борту Landsat 7 вышел из строя прибор <<Scan Line Corrector>> (SLC), что существенным образом снизило объем полезной информации, получаемой со спутника. На практике, поломка SLC выразилась в <<располосованности>> снимков, получаемых от ETM+.

	Таким образом, до тех пор, пока не будет запущен новый спутник группировки Landsat (запуск которого намечен на 2012-ый год) и данные от него не начнут поступать в открытый доступ, наиболее эффективным представляется использованние данных ДЗЗ от камеры TM КА Landsat 5.

	Данные, полученные камерами TM и ETM+, распространяются в виде каталогов, в которых наличиствуют, как минимум, 7 (TM) или 8 (ETM+) файлов формата GeoTIFF. Каждый из файлов формата GeoTIFF хранит соответствующий канал снимка в виде 8-ми битного беззнакового одноканального растра. В файлах сохранена также информация о геопривязке снимка и о его проекции (система координат - WGS84, проекция UTM). Перед распространением снимки подвергаются предварительной обработке с целью коррекции некоторых типов искажений.

	Распространение данных, полученных от камер TM и ETM+ - свободное и бесплатное, однако для получения некоторых кадров (не обязательно самых свежих) требуется предзаказ снимков и ожидание подготовки снимка к загрузке в течении некоторого периода. Большинство снимков, впрочем, доступны для скачивания сразу же;

	\item Матрицы высот, предоставляемые проектами Aster GDEM и SRTM. \cite{gis-lab-aster} \cite{gis-lab-srtm}

	Проекты Aster GDEM и SRTM предоставляют матрицы высот земной поверхности, позволяющие решать различные задачи картографирования, требующие знание абсолютных высот различных точек земной поверхности - к таковым задачам относятся, например, нанесение изолиний высот на топографические карты или составление трехмерных моделей рельефа;

	\item Продукты, полученные на основе данных ДЗЗ, получаемых электронно - оптической системой MODIS. \cite{gis-lab-modis}

	Камеры MODIS установлены на КА Aqua и КА Terra.

	Камеры MODIS являются мультиспектральными камерами - они выполняют съемку в 36-ти спектральных каналах (разрешение от 250-ти метров на пиксель до 1000 метров на пиксель). На основе данных, получаемых камерами MODIS, расчитывается ряд продуктов, могущих быть использоваными для решения самых разнообразных геоинформационных задач - например, для решения задачи слежения за пожарами может быть использован продукт MOD14 (карта тепловых аномалий), а для решения задач наблюдения за растительным покровом может быть использован продукт MOD13 (карта индексов растительности).

\end{itemize}

\newcommand{\ee}{<<EarthExplorer>>\xspace}

\myparagraph{Получение данных ДЗЗ с помощью web - сервиса \ee}

Существует несколько способов получения свободно распространяемых данных ДЗЗ - наиболее удобным является использование web - сервиса \ee \cite{earthexplorer}, разработанного и поддерживаемого Геологической службой США (United States Geological Survey; USGS) и доступного по адресу \url{http://earthexplorer.usgs.gov}.

\mysubparagraph{Создание учетной записи}

Получение данных ДЗЗ с помощью сервиса \ee возможно после создания учетной записи на нем, для чего необходимо выполнить следующие действия:

\begin{enumerate}

	\item Запустить web - броузер.

	{\bf Для доступа к \ee настоятельно рекомендуется использовать web - броузер Firefox};

	\item Перейти на главную страницу сервиса;
	\item Нажать кнопку <<Register>> панели инструментов сервиса (рисунок \ref{image:1:418});
	\item Заполнить форму <<1. Login>> (рисунок \ref{image:1:420}), после чего необходимо нажать кнопку \linebreak <<Continue>> данной формы;
	\item Заполнить форму <<2. User Affiliation>> данными, приведенными на рисунке \ref{image:1:422}, после чего необходимо нажать кнопку <<Continue>> данной формы;
	\item Заполнить форму <<3. Address>> данными (пример заполнения формы приведен на рисунке \ref{image:1:424} - значения полей <<First Name>>, <<Last Name>> и <<E-mail>> должны, очевидно, отличаться от приведенных), после чего необходимо нажать кнопку <<Continue>> данной формы;
	\item После успешного заполнения всех трех форм, необходимых для регистрации, новая учетная запись будет создана и автоматически будет выполнен вход в сервис.

\end{enumerate}

Подтверждения регистрации по электронной почте не требуется.

\begin{landscape}
\mimage{1:418}{418}{Кнопка <<Register>> панели инструментов сервиса \ee}{width = 0.9\linewidth}
\mimage{1:420}{420}{Форма <<1. Login>>}{width = 0.9\linewidth}
\mimage{1:422}{422}{Форма <<2. User Affiliation>>}{width = 0.9\linewidth}
\mimage{1:424}{424}{Форма <<3. Address>>}{width = 0.9\linewidth}
\end{landscape}

\mysubparagraph{Вход и выход}

Вход в сервис \ee под имеющейся учетной запись выполняется с помощью соответствующей формы, открываемой нажатием на кнопку <<Login>> панели инструментов сервиса.

Для выхода из сервиса необходимо нажать кнопку <<Logout>> панели инструментов сервиса.

\mysubparagraph{Получение данных ДЗЗ}

Для получения данных ДЗЗ с помощью сервиса \ee необходимо выполнить следующие действия:

\begin{enumerate}

	\item Войти в сервис \ee;
	\item Перейти на главную страницу сервиса \ee, для чего необходимо нажать кнопку <<Home>> панели инструментов сервиса (главная страница сервиса приведена на рисунке \ref{image:1:505});
	\item Выполнить поиск интересующего спутникового снимка, для чего необходимо выполнить следующие действия:

		\begin{enumerate}

			\item Во вкладке <<Search Criteria>> с помощью поля <<Search Criteria Summary>>, предоставляемого проектом <<Google Maps>>, перейти к интересующей области земной поверхности (рисунок \ref{image:1:505});
			\item Нажать кнопку <<Use Map>> поля <<Coordinates>> вкладки <<Search Criteria>>, после чего вся область земной поверхности, снимок которой отображен в поле <<Search Criteria Summary>>, будет выделена (рисунок \ref{image:1:506});
			\item Увеличить масштаб спутникового снимка, отображенного в поле <<Search Criteria Summary>> (рисунок \ref{image:1:507});
			\item Уточнить границы интересующей области земной поверхности перетаскиванием меток 1, 2, 3 и 4 в поле <<Search Criteria Summary>> (рисунок \ref{image:1:507});
			\item Выбрать интересующий временной диапазон, для чего необходимо ввести дату начала и дату конца диапазона в поле <<Date Range>> вкладки <<Search Criteria>> (рисунок \ref{image:1:507});
			\item Перейти к выбору источника данных ДЗЗ, для чего необходимо нажать кнопку <<Data Sets>> вкладки <<Search Criteria>>;
			\item Выбрать интересующие источники данных ДЗЗ, для чего необходимо отметить поля, соответствующие означенным источникам, в дереве вкладки <<Data Sets>> (рисунок \ref{image:1:508}).

			Для выбора камеры TM КА Landsat 5 необходимо отметить поле <<L4-5 TM>> узла <<Landsat Archive>>.

			Для выбора камеры ETM+ КА Landsat 7 необходимо отметить поле <<L7 ETM+ SLC-off (2003 - present)>> (для получения снимков, сделанных после выхода из строя SLC) или поле <<L7 ETM+ SLC-on (1999 - 2003)>> (для получения снимков, сделанных до выхода из строя SLC) узла <<Landsat Archive>>;

			\item Перейти к выбору спутникового снимка, для чего необходимо нажать кнопку \linebreak <<Results>> вкладки <<Data Sets>>;

		\end{enumerate}

	\item Выбрать и загрузить интересующий спутниковый снимок из списка снимков, соответствующих заданным критериям, для чего необходимо выполнить следующие действия:

		\begin{enumerate}

			\item Во вкладке <<Result>> найти интересующий спутниковый снимок (рисунок \ref{image:1:513});
			\item Получить область покрытия спутникового снимка, для чего необходимо нажать кнопку <<Show Footprint>> (рисунок \ref{image:1:514});
			\item Получить превью спутникового снимка, для чего необходимо нажать кнопку <<Show Browse Overlay>> (рисунок \ref{image:1:515});
			\item Получить характеристики спутникового снимка, для чего необходимо нажать кнопку <<Show Metadata and Browse>>, после чего характеристики будут выведены в дополнительном окне (рисунок \ref{image:1:516});
			\item Загрузить спутниковый снимок, для чего необходимо выполнить следующие действия:

			\begin{enumerate}

				\item Нажать кнопку <<Download Options>>;
				\item В окне <<Download Options>> выбрать пункт <<Level 1 Product>> (только для спутниковых снимков, сделанных камерами TM и ETM+; рисунок \ref{image:1:517}), после чего необходимо нажать кнопку <<Select Download Option>>;
				\item В окне <<Download Scene>> нажать кнопку <<Download>> (рисунок \ref{image:1:518}), после чего начнется загрузка интересующего спутникового снимка.

			\end{enumerate}

		\end{enumerate}

\end{enumerate}

Сервис \ee распространяет спутниковые снимки, сделанные камерами TM и ETM+, в виде сжатого\footnote{Сжатие выполняется утилитой gzip.} архива\footnote{Архивация выполняется утилитой tar.}, содержимое которого может быть извлечено большинством современных архиваторов (например, утилитами tar и gunzip в ОС семейства Unix или утилитой 7 - Zip в ОС GNU/Linux и ОС семейства Windows).

В случае камеры TM в архиве находятся несколько файлов, ключевыми из которых являются файлы формата GeoTIFF, хранящие каналы (B10 - B70) спутникового снимка. Файлы, хранящие каналы, имеют одинаковые названия за исключением индекса канала.

\begin{landscape}
\mimage{1:505}{505}{Интересующая область земной поверхности}{width = 0.9\linewidth}
\mimage{1:506}{506}{Выбор интересующей области земной поверхности}{width = 0.9\linewidth}
\mimage{1:507}{507}{Критерии поиска спутникового снимка}{width = 0.9\linewidth}
\mimage{1:508}{508}{Выбор камер TM КА Landsat 4 и Landsat 5 в качестве источников данных ДЗЗ}{width = 0.9\linewidth}
\mimage{1:513}{513}{Результаты поиска}{width = 0.9\linewidth}
\mimage{1:514}{514}{Область покрытия спутникового снимка}{width = 0.9\linewidth}
\mimage{1:515}{515}{Превью спутникового снимка}{width = 0.9\linewidth}
\mimage{1:516}{516}{Характеристики спутникового снимка}{width = 0.9\linewidth}
\mimage{1:517}{517}{Окно <<Download Options>>}{width = 0.9\linewidth}
\mimage{1:518}{518}{Окно <<Download Scene>>}{width = 0.9\linewidth}
\end{landscape}

\mysubsubsection{Свободный инструментарий решения геоинформационных задач}

Мир свободного открытого программного обеспечения предлагает широкий спектр инструментов решения различных геоинформационных задач.

Ключевыми из означенных инструментов являются:

\begin{itemize}

	\item Программная библиотека GDAL / OGR (Geospatial Data Abstraction Library / OGR Simple Feature Library) \cite{gdal-ogr} языков программирования C, C++ и Python;
	\item Геоинформационная система (ГИС) Quantum GIS \cite{qgis};
	\item ГИС Grass \cite{grass}.

\end{itemize}

Перечисленные инструменты являются кроссплатформенными и могут быть использованы в различных ОС на различных аппаратных платформах - так, перечисленные инструменты могут быть использованы в ОС GNU/Linux, ОС семейства BSD и ОС семейства Windows.

Для ОС семейства Windows наиболее удобным способом установки свободных открытых инструментов для решения геоинформационных задач является использования установщика, разрабатываемого в рамках проекта OSGeo4W \cite{osgeo4w}.

\myparagraph{Программный комплекс OSGeo4W}

Для установки свободных открытых инструментов решения геоинформационных задач с помощью установщика OSGeo4W необходимо выполнить следующие действия:

\begin{enumerate}

	\item Загрузить исполняемый файл установщика с официального сайта проекта OSGeo4W \cite{osgeo4w} или получить установщик иным способом;
	\item Запустить на выполнение исполняемый файл установщика;
	\item В окне <<Установка OSGeo4W>> выбрать тип установки (рекомендуется выбрать <<Расширенная установка>>, что проиллюстрировано рисунком \ref{image:1:1}), после чего необходимо нажать кнопку <<Далее>>;
	\item В окне <<Установка OSGeo4W - Выберите режим инсталляции>> выбрать способ получения пакетов с устанавливаемым инструментарием (рисунки \ref{image:1:2} и \ref{image:1:21}), после чего необходимо нажать кнопку <<Далее>>.

	\begin{itemize}

		\item Пункт <<Установить из интернет>> предполагает загрузку всех пакетов из сети Интернет (рекомендуемый выбор - в этом случае будут установлены самые свежие версии пакетов);
		\item Пункт <<Установить из локального каталога>> предполагает использование ранее загруженных пакетов, сохраненных в специальном каталоге - кэше пакетов;

	\end{itemize}

	\item В окне <<Установка OSGeo4W - Выберите целевой каталог установки>> выбрать каталог, в который будут установлены интересующие геоинформационные инструменты (рисунок \ref{image:1:3}), после чего необходимо нажать кнопку <<Далее>>;
	\item В окне <<Установка OSGeo4W - Выберите локальный каталог пакетов>> выбрать каталог, в котором будет будет организован или уже находится кэш пакетов (рисунок \ref{image:1:4}), после чего необходимо нажать кнопку <<Далее>>;
	\item В окне <<Установка OSGeo4W - Выберите тип интернет - соединения>> указать параметры интернет-соединения (рисунок \ref{image:1:5}), после чего необходимо нажать кнопку <<Далее>>.

	Данный пункт выполняется только в случае установки инструментария с загрузкой пакетов из сети Интернет;

	\item В окне <<Установка OSGeo4W - Выберите пакеты>> выбрать те геоинформационные инструменты, которые необходимо установить, после чего необходимо нажать кнопку <<Далее>>.

	В рамках настоящей лабораторной работы необходимо установить следующие инструменты:

	\begin{itemize}

		\item Программная библиотека GDAL / OGR - пункт <<gdal: The GDAL/OGR library and commandline tools>> узла <<Libs>> (рисунок \ref{image:1:7});
		\item ГИС Quantum GIS - пункт <<qgis: Quantum GIS (desktop)>> узла <<Desktop>> (рисунок \ref{image:1:6}).

	\end{itemize}

	{\bf Для каждого из инструментов необходимо ставить метку в столбце <<Bin>>;}

	\item После того, как устанавливаемые инструменты были выбраны, начнется процесс установки инструментов, проиллюстрированный рисунками \ref{image:1:8} и \ref{image:1:9};
	\item По окончанию установки в окне <<Установка OSGeo4W - Состояние установки и создание ярлыков>> необходимо отметить те средства быстрого доступа к установленному инструментарию, которые требуется создать (рисунок \ref{image:1:10}).

	В окне <<Установка OSGeo4W - Состояние установки и создание ярлыков>> необходимо отметить пункт <<Добавить ярлык в меню Пуск>>;

	\item Завершить установку нажатием кнопки <<Готово>> окна <<Установка OSGeo4W>>;
	\item Добавить в переменную <<Path>> окружения путь к подкаталогу bin каталога установки инструментария\footnote{В каталоге bin расположены файлы динамических библиотек библиотеки GDAL / OGR - ОС должна знать, где файлы динамических библиотек можно найти.}, для чего необходимо выполнить следующие действия:

	\begin{enumerate}

		\item Открыть окно <<Панель управления>>;
		\item Открыть окно <<Свойства системы>>;
		\item В окне <<Свойства системы>> выбрать вкладку <<Дополнительно>> (рисунок \ref{image:1:101}), после чего необходимо нажать кнопку <<Переменные среды>>;
		\item В поле <<Системные переменные>> окна <<Переменные среды>> выбрать переменную <<Path>> (рисунок \ref{image:1:102}), после чего необходимо нажать кнопку <<Изменить>>;
		\item В окне <<Изменение системной переменной>> с помощью поля <<Значение переменной>> дописать в конец переменной <<Path>> через точку с запятой полный путь до подкаталога bin каталога, в который был установлен геоинформационный инструментарий (рисунок \ref{image:1:103}), после чего необходимо нажать кнопку <<OK>>.

	\end{enumerate}

\end{enumerate}

\mimage{1:1}{1}{Окно <<Установка OSGeo4W>>}{}
\mimage{1:2}{2}{Окно <<Установка OSGeo4W - Выберите режим инсталляции>> - загрузка пакетов из сети Интернет}{}
\mimage{1:21}{21}{Окно <<Установка OSGeo4W - Выберите режим инсталляции>> - использование кэша пакетов}{}
\mimage{1:3}{3}{Окно <<Установка OSGeo4W - Выберите целевой каталог установки>>}{}
\mimage{1:4}{4}{Окно <<Установка OSGeo4W - Выберите локальный каталог пакетов>>}{}
\mimage{1:5}{5}{Окно <<Установка OSGeo4W - Выберите тип интернет - соединения>>}{}
\begin{landscape}
\mimage{1:7}{7}{Установка программной библиотеки GDAL / OGR}{width = 0.94\linewidth}
\mimage{1:6}{6}{Установка ГИС Quantum GIS}{width = 0.94\linewidth}
\mimage{1:8}{8}{Процесс установки (начало)}{width = 0.94\linewidth}
\mimage{1:9}{9}{Процесс установки (окончание)}{width = 0.94\linewidth}
\mimage{1:10}{10}{Окно <<Установка OSGeo4W - Состояние установки и создание ярлыков>>}{width = 0.94\linewidth}
\end{landscape}
\mimage{1:101}{101}{Окно <<Свойства системы>> - вкладка <<Дополнительно>>}{width = \linewidth}
\mimage{1:102}{102}{Окно <<Переменные среды>>}{}
\mimage{1:103}{103}{Окно <<Изменение системной переменной>>}{}

\myparagraph{Программная библиотека GDAL / OGR}

Программная библиотека GDAL / OGR (Geospatial Data Abstraction Library / OGR Simple Feature Library) \cite{gdal-ogr} является свободным открытым инструментарием для простейшей обработки геоинформационных данных.

Библиотека GDAL / OGR разработана на языках программирования C и C++ и обладает официально поддерживаемыми интерфейсами для языков программирования C, C++ и Python.

Библиотека GDAL / OGR позволяет:

\begin{itemize}

	\item Создавать, загружать и сохранять растровые изображения (растровые слои), сохраненные в файлах различных форматов (GeoTIFF, Erdas Imagine, JPEG, ECW и прочие, используемые в ходе решения геоинформационных задач);
	\item Создавать, загружать и сохранять векторные изображения (векторные слои), сохраненные в файлах различных форматов (ESRI Shape, PostGIS, MapInfo и прочие, используемые в ходе решения геоинформационных задач);
	\item Выполнять привязку слоев к целевой системе координат и проекцию слоев в требуемую проекцию;
	\item Создавать, сохранять и выполнять различного рода обработку привязок данных к системам координат;
	\item Репроецировать данные из одной проекции в другую;
	\item Работать с географическими координатами;
	\item Растеризовывать векторные слои и векторизовывать растровые;
	\item Выполнять базовую обработку растровых и векторных слоев (например, рассчитывать гистограммы растровых слоев);
	\item Выполнять прочие действия по обработке растровых и векторных слоев.

\end{itemize}

\mysubparagraph{Подключение библиотеки к проекту}

Для подключения библиотеки GDAL / OGR к проекту, разрабатываемому на языке программирования C++ с помощью интегрированной среды разработки (Integrated Development Environment; IDE) Visual Studio\footnote{Используется Visual Studio 2005, однако приведенный алгоритм справедлив и для других версий Visual Studio.}, необходимо выполнить следующие действия:

\begin{enumerate}

	\item Открыть окно свойств проекта;
	\item В поле <<Configuration>> окна свойств проекта выбрать пункт <<All Configurations>> (рисунок \ref{image:1:201});
	\item В поле <<Additional Include Directories>> вкладки <<Configuration Properties / C, C++ / General>> указать полный путь до каталога, в котором расположены заголовочные файлы библиотеки GDAL / OGR (данным каталогом является подкаталог include каталога, в который была выполнена установка инструментария; рисунок \ref{image:1:202});
	\item В поле <<Additional Library Directories>> вкладки <<Configuration Properties / Linker / General>> (рисунок \ref{image:1:203}) указать через точку с запятой полные пути до следующих каталогов:

	\begin{itemize}

		\item Каталог, содержащий файлы динамических библиотек библиотеки GDAL / OGR (данным каталогом является подкаталог bin каталога, в который была выполнена установка инструментария);
		\item Каталог, содержащий файлы статических библиотек библиотеки GDAL / OGR, необходимые для связывания проекта с соответствующими динамическими библиотеками (данным каталогом является подкаталог lib каталога, в который была выполнена установка инструментария);

	\end{itemize}

	\item В поле <<Additional Dependencies>> вкладки <<Configuration Properties / Linker / Input>> указать <<gdal\_i.lib>>, что является указанием компилятору связать исполняемый файл проекта с динамическими библиотеками библиотеки GDAL / OGR (рисунок \ref{image:1:204}).

\end{enumerate}

\begin{landscape}
\mimage{1:201}{201}{Поле <<Configuration>>}{width = 0.94\linewidth}
\mimage{1:202}{202}{Поле <<Additional Include Directories>>}{width = 0.94\linewidth}
\mimage{1:203}{203}{Поле <<Additional Library Directories>>}{width = 0.94\linewidth}
\mimage{1:204}{204}{Поле <<Additional Dependencies>>}{width = 0.94\linewidth}
\end{landscape}

\mysubparagraph{Функционал библиотеки}

Программная библиотека GDAL / OGR предоставляет самый разнообразный функционал обработки данных ДЗЗ и векторных слоев электронных карт, часть которого будет рассмотрена в настоящей лабораторной работе.

Для получения справочной информации о библиотеке можно воспользоваться online - документацией, доступной по ссылке \cite{gdal-ogr}.

Здесь и далее предполагается, что разработка прикладной программы ведется на языке программирования C++.

Для чтения изображения из файла формата GeoTIFF необходимо выполнить следующие действия:

\begin{enumerate}

	\item Загрузить драйвера растровых форматов файлов, для чего необходимо вызвать функцию \verb|GDALAllRegister()|;
	\item Открыть на чтение целевой файл, для чего необходимо вызвать функцию \verb|GDALOpen()|, прототип которой приведен в листинге \ref{listing:1:gdalopen} (первая строка).

	\mylistingbegin{1:gdalopen}{Открытие на чтение целевого файла с помощью функции GDALOpen()}
	\begin{lstlisting}

	GDALDatasetH GDALOpen(const char * pszFilename, GA_ReadOnly);

	GDALDataset * src = (GDALDataset *) GDALOpen(pszFilename, GA_ReadOnly);

	\end{lstlisting}
	\mylistingend

	Параметр \verb|pszFilename| функции \verb|GDALOpen()| должен содержать путь и имя целевого файла.

	Функция возвращает указатель на описатель файла (объект класса \verb|GDALDataset|) в случае его успешного открытия (указатель требуется привести к типу \verb|GDALDataset *|, вторая строка листинга \ref{listing:1:gdalopen}) или \verb|NULL|, если файл открыть не удалось;

	\item Считать из файла целевые каналы.

	Для чтения из файла целевых каналов может быть использован метод \verb|RasterIO()| класса \verb|GDALDataset|. Формат вызова указанного метода для чтения одного 8-ми битного беззнакового канала приведен в листинге \ref{listing:1:rasterio:read}.

	\mylistingbegin{1:rasterio:read}{Чтение 8-ми битного беззнакового канала с помощью метода RasterIO() класса GDALDataset}
	\begin{lstlisting}

	src->RasterIO(GF_Read, 0, 0, width, height, buf, width, height, GDT_Byte, 1, NULL, 0, 0, 0);

	\end{lstlisting}
	\mylistingend

	В листинге \ref{listing:1:rasterio:read}:

	\begin{itemize}

		\item \verb|width|, \verb|height| - длина строки изображения (в пикселях), количество строк в изображении;
		\item \verb|buf| - указатель на буфер размером \verb|width * height| байт, в который будет загружен целевой канал (тип \verb|buf|, очевидно, \verb|unsigned char *|).

	\end{itemize}

	Длина строки изображения и количество строк в изображении могут быть получены с помощью методов \verb|GetRasterXSize()| и \verb|GetRasterYSize()| соответственно.

	Содержимое канала развертывается в буфер \verb|buf| построчно - так, яркость пикселя, расположенного в $i$-ой строке, $j$-ом столбце изображения, сохранена в $(i * width + j)$-ом элементе буфера \verb|buf| при условии нумерации строк и столбцов от нуля.

	Рассматриваемый метод возвращает значение константы \verb|CE_None| компилятора в случае, если чтение было выполнено успешно, и значение константы \verb|CE_Failure| компилятора в противном случае;

	\item Закрыть файл, для чего можно воспользоваться функцией \verb|GDALClose()|, прототип которой приведен в листинге \ref{listing:1:gdalclose}. 

	\mylistingbegin{1:gdalclose}{Закрытие файла с помощью функции GDALClose()}
	\begin{lstlisting}

	void GDALClose (GDALDatasetH hDS);

	\end{lstlisting}
	\mylistingend

	Единственный параметр функции \verb|GDALClose()| суть есть указатель на описатель файла (указатель на объект класса \verb|GDALDataset|).

	В ряде случаев вызов функции \verb|GDALClose()| эквивалентен выполнению оператора \verb|delete|, однако {\bf для уничтожения объекта класса \verb|GDALDataset| рекомендуется всегда использовать функцию \verb|GDALClose()|}.

\end{enumerate}

Для сохранения нового растра в файле формата GeoTIFF необходимо выполнить следующие действия:

\begin{enumerate}

	\item Получить указатель на описатель драйвера формата GeoTIFF, для чего необходимо воспользоваться методом \verb|GetDriverByName()| менеджера драйверов растровых форматов.
		
	Формат вызова метода \verb|GetDriverByName()| для получения указателя на описатель драйвера формата GeoTIFF приведен в листинге \ref{listing:1:getdriverbyname};

	\mylistingbegin{1:getdriverbyname}{Получение указателя на описатель драйвера формата GeoTIFF с помощью метода GetDriverByName() класса GDALDriverManager}
	\begin{lstlisting}

	GDALDriver * drv = GetGDALDriverManager()->GetDriverByName("GTiff");

	\end{lstlisting}
	\mylistingend

	\item Создать новое изображение и сохранить его в файле формата GeoTIFF, для чего необходимо воспользоваться методом \verb|Create()| описателя драйвера формата GeoTIFF.

	Формат вызова указанного метода для создания изображения с 8-ми битными беззнаковыми каналами приведен в листинге \ref{listing:1:create}.

	\mylistingbegin{1:create}{Создание и сохранение нового изображения}
	\begin{lstlisting}

	GDALDataset * dst = drv->Create(fname, width, height, nBands, GDT_Byte, NULL);

	\end{lstlisting}
	\mylistingend

	В листинге \ref{listing:1:create}:

	\begin{itemize}

		\item \verb|fname| - путь и имя файла формата GeoTIFF, в который будет сохранено изображение;
		\item \verb|width| и \verb|height| - длина строки изображения (в пикселях) и количество строк в изображении соответственно;
		\item \verb|nBands| - количество каналов в изображении (1 или 3).

	\end{itemize}

	Рассматриваемый метод возвращает указатель на описатель файла в случае успешного создания нового изображения и \verb|NULL| в случае, если создание изображения завершилось неудачей;

	\item Сохранить спектральные каналы изображения.

	Для сохранения в файл целевых каналов изображения может быть использован метод \verb|RasterIO()| класса \verb|GDALDataset|. Формат вызова указанного метода для сохранения 8-ми битного беззнакового канала приведен в листинге \ref{listing:1:rasterio:write}.

	\mylistingbegin{1:rasterio:write}{Сохранение 8-ми битных беззнаковых каналов - метод RasterIO() класса GDALDataset}
	\begin{lstlisting}

	dst->RasterIO(GF_Write, 0, 0, width, height, buf, width, height, GDT_Byte, 1, & band, 0, 0, 0);

	\end{lstlisting}
	\mylistingend

	В листинге \ref{listing:1:rasterio:write}:

	\begin{itemize}

		\item \verb|width|, \verb|height| - длина строки изображения (в пикселях) и количество строк в изображении соответственно;
		\item \verb|buf| - указатель на буфер размером \verb|width * height| байт, в котором расположен целевой канал (тип \verb|buf|, очевидно, \verb|unsigned char|);
		\item \verb|band| - номер сохраняемого канала (каналы нумеруются от единицы; переменная \verb|band| имеет тип \verb|int|).

	\end{itemize}
			
	Рассматриваемый метод возвращает значение константы \verb|CE_None| компилятора в случае, если запись была выполнена успешно, и значение константы \verb|CE_Failure| компилятора в противном случае;

	\item Скопировать привязку к системе координат и данные о проекции исходного изображения (опционально).

	В случае, если результирующее изображение суть есть производное от некоторого (некоторых) исходного (исходных), то перед закрытием файла с результирующим изображением необходимо сохранить в данный файл данные о привязке изображения к системе координат и о проекции изображения, для чего можно воспользоваться методами \linebreak \verb|GetGeoTransform(), SetGeoTransform(), GetProjectionRef(), SetProjection()| (в листинге \ref{listing:1:copygpp} приведен пример означенного копирования для изображений, производных от данных ДЗЗ, полученных камерами TM и ETM+ и загруженных с помощью сервиса \ee).

	\mylistingbegin{1:copygpp}{Копирование привязки к системе координат и данных о проекции}
	\begin{lstlisting}

	double transform_coef[6];

	src->GetGeoTransform(transform_coef);
	dst->SetGeoTransform(transform_coef);

	dst->SetProjection(src->GetProjectionRef());

	\end{lstlisting}
	\mylistingend

	В листинге \ref{listing:1:copygpp}:

	\begin{itemize}

		\item \verb|src| - описатель файла с исходным изображением;
		\item \verb|dst| - описатель файла с результирующим изображением;

	\end{itemize}

	\item Закрыть файл.

\end{enumerate}

Для использования всех вышеперечисленных классов и функций необходимо подключить к программе следующие заголовочные файлы:

\begin{itemize}

	\item gdal\_priv.h;
	\item cpl\_conv.h.

\end{itemize}

\mysubparagraph{Пример использования библиотеки}

В листинге \ref{listing:1:example} приведен исходный код программы, разработанной на языке программирования C++ и выполняющей комплексирование зеленого, красного и ближнего инфракрасного каналов части спутникового снимка, сделанного камерой TM, с сохранением информации о привязке и проекции снимка.

Результат выполнения программы приведен на рисунке \ref{image:1:303} - результирующее изображение совмещено с каналом B50 средствами ГИС Quantum GIS.

\mysource{trash/1/main.cpp}{1:example}{Пример использования библиотеки GDAL / OGR}

\begin{landscape}
\mimage{1:303}{303}{Результат выполнения программы}{width = 0.94\linewidth}
\end{landscape}

